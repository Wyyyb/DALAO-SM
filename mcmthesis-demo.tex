\documentclass{mcmthesis}
\mcmsetup{CTeX = false,   % 使用 CTeX 套装时,设置为 true
	tcn = 78435, problem = E,
	sheet = true, titleinsheet = true, keywordsinsheet = true,
	titlepage = false, abstract = true}
\usepackage{palatino}
\usepackage{float}
\usepackage{etoc}
\usepackage{array}
\usepackage{setspace}
\newcolumntype{I}{!{\vrule width 3pt}}
\newlength\savedwidth
\newcommand\whline{\noalign{\global\savedwidth\arrayrulewidth
		\global\arrayrulewidth 1.2pt}%
	\hline
	\noalign{\global\arrayrulewidth\savedwidth}}
\newlength\savewidth
\newcommand\shline{\noalign{\global\savewidth\arrayrulewidth
		\global\arrayrulewidth 1.2pt}%
	\hline
	\noalign{\global\arrayrulewidth\savewidth}}
\title{How does climate change influence regional instability?}
\author{}
\date{}
\begin{document}
	\begin{abstract}
		In the present airport construction, the reliability and efficiency of security inspection are of vital significance, while the occurrence of congestion in checkpoints is undeniably frequent. This report aims to analyze the current process of airport security checkpoints and propose rational modifications.
		
		Firstly, we do the parameter analysis and process division based on the current situation and data. We assume that the flow of passengers obeys \textbf{PoissonDistribution}, and that the checking time obeys \textbf{Exponential Distribution}. According to the data given, we estimate related parameters using \textbf{Moment Estimation}, and obtain the quantitative model. Meanwhile, we perform the integral analysis of the system and figure out the relationship between the maximal flow of passengers and the proportion of ID-Check entrance number to screening lane number. After the simulation, we conclude that the bottleneck appears owing to screening lanes.
		
		Secondly, we propose two modifications to our model. According to the time consumed in different areas, we figure out the ideal proportion of the number of entrances and lanes so as to reach the highest throughput. Applying the M/M/S model in \textbf{Queuing Theory}, we obtain several \textbf{dynamic optimization} functions based on the principle of minimizing the fluctuation of the average waiting time. Considering both mentalities of passengers and construction costs of airports, we attain the final optimization. 
		
		Thirdly, we do \textbf{sensitivity analysis} of the modified model. In previous analysis, we assume that passengers all obey rules and act normally, but due to discrepancies of cultural environment, passengers tend to behavior diversely, which could impact our model. Therefore, we mainly analyze three types of passengers and impacts of their behaviors on our model. Passengers who are queue-jumpers may cause disputes leading to the paralysis of specific entrances, and those who dress special may prolong the screening time. Furthermore, ones who hold obscure perception of contraband may be more prone to get additional screening.
		
		Finally, we propose policy and recommendations for security managers based on our optimal model. For instance, rearrange the number of entrances or lanes according to the current throughput, establish a baggage buffer at the end of X-ray belts, and promote Pre-Check.
		
		
		
		\begin{keywords}
			Parameter Estimation; Queuing Theory; Dynamic Optimization; Maximal Throughput; Airport Security; Cultural Diversity
		\end{keywords}
	\end{abstract}
	\maketitle
	
	\newpage
	\begin{center}
		\tableofcontents
		\setcounter{page}{0}
		\thispagestyle{empty}
	\end{center}
	\newpage
	
	\section{Introduction}
	
	A \textbf{fragile state} is a low-income country or a sovereign state that is characterized by weak state capacity and/or weak state legitimacy[1]. It increases the vulnerability of citizens towards a range of shocks, including economic fluctuation, political upheaval, and so forth. 
	
	Particularly, \textbf{climate shock} such as unpredictable natural disasters, extreme weather, decreasing cultivated land and changing ranges of plants and animals may dramatically aggravate fragile states and may further leads to regional violent conflicts when social fragmentation and weak governance exist as well. Therefore, it is of vital significance to research on how climate change will affect the instability or fragility of a state and how human intervention can mitigate such negative impact.
	
	In this paper, we mainly focus on analyzing the influence of climate change on regional instability and fragility. More specifically, we pay attention to the following issues:
	\begin{itemize}
		\item Develop a model that determines the fragility of a country and identify when a state is fragile, vulnerable, or stable. Simultaneously, the model will measure the impact of climate change and analyze direct and indirect means by which climate change increases the fragility.
		\item Select A, one of the top 10 most fragile states as determined by the Fragile State Index (http://fundforpeace.org/fsi/data/) and use the model to analyze how climate change has increased the fragility of it and show in what way(s) A may be less fragile without these effects.
		\item Apply the model on B to measure its fragility, and determine in what way and when climate change may increase the fragility of the state. Identify any definitive indicators. Define a tipping point and predict when B may reach it.
		\item Use the model to show which state driven interventions could alleviate the risk of climate change and prevent a country from becoming a fragile state. Explain the effect of human intervention and predict the total cost of intervention for this country.
		\item Analyze whether the model will correctly work on smaller “states” (such as cities) or larger “states” (such as continents) and modify the models accordingly.
		\item Assess strengths and weaknesses of the model.
	\end{itemize}
	
	In the following chapters, we will demonstrate and illustrate our model in details, as well as evaluating the model in all directions.
	
	\section{Assumption}
	
	\begin{itemize}
		\item Assume that climate change is random and cannot be controlled by human.
		
		\item Assume that the variation of economy, society and policy factors in a country is only dependent on current country status if there is no climate change.
		
		\item Assume that there is no sudden revolution in the country analyzed.
		
		\item Assume the time delay between climate change and national response is short enough.
		
		\item Assume that all countries have the same evaluation function and prediction function.
		
	\end{itemize}
	
	\section{A common model for most countries}
	In this section, we draw a relationship schema of several vital factors to show the direct influences between them. Take economic policy as an example, it has a strong direct effect on GDP under most circumstances. And in the schema, we can also see the indirect influence since all the boxes are connected by arrow lines.
	
	After that, an evaluation function can be determined by weighting the factors’ contribution to fragility. Even more, we can also gain a recursion function to predict the country's fragility in a few years. Using these functions to evaluate current status of some typical countries, an evaluation standard then be settled.
	
	\subsection{Model establishing without climate change}
	Based on the schema below (\textbf{Figure 1}), we focus on several factors that might have the most impact on a country's fragility. Policies issued by government, military activities, social problems and economic crisis are four major factors effecting fragility. These factors also have interconnections, which makes this model more realistic and manageable.
	
	\begin{figure}[h]
		\small
		\centering
		\includegraphics[width=14cm]{figure2.png}
		\caption{Stage-structured model of current process} \label{fig:Stage-structured model of current process}
	\end{figure}
	
	\noindent Additionally, the quantities involved in the function are shown in \textbf{Table 1} and \textbf{Table 2}. We list subscripts and primary notations apart.\\
	
	\begin{table}[htbp]
		\renewcommand\arraystretch{1.5}
		\footnotesize
		\centering
		\begin{tabular}{m{3cm}<{\centering}|m{10cm}<{\centering}}
			\whline
			\textbf{Subscript}&\textbf{Definition}\\
			\whline 
			$N$&$N_{th}$ year\\ 
			\shline
		\end{tabular}
		\caption{Definition of Subscripts in the model}\label{tab:Definition of Subscripts in the model}
	\end{table}
	\begin{table}[htbp]
		\renewcommand\arraystretch{1.5}
		\footnotesize
		\centering
		\begin{tabular}{m{3cm}<{\centering}|m{10cm}<{\centering}}
			\whline
			\textbf{Notation}&\textbf{Definition}\\
			\whline 
			$F$&Degree of Fragility\\
			$C$&Climate factor\\
			$E$&Economic factor\\
			$S$&Social factor\\
			$P$&Policy factor\\ 
			$M$&Military factor\\
			$X$&External Intervention\\
			\shline
		\end{tabular}
		\caption{Definition of notations in the model}\label{tab:Definition of notations in the model}
	\end{table}
	
	To evaluate the fragility of a country, we define the evaluation function $F$
	
	\begin{equation}
	F = w 
	\left(
	\begin{matrix}
	P_N \\ M_N \\ E_N \\ S_N
	\end{matrix}
	\right)
	\end{equation}
	
	In the function, $w$ is the weight of the variables, which is $(w_1, w_2, w_3, w_4)$. Since in the subsequent paragraphs we will normalize $P_N, M_N, E_N, S_N$, and for conciseness of the formula, we could assign w to be 1 for each variable.
	
	As assumed, if climate and external intervention remains zero, our major factors should be a \textbf{Markov Model}, which is
	
	$$
	\left(
	\begin{matrix}
	P_{N+1} \\ M_{N+1} \\ E_{N+1} \\ S_{N+1}
	\end{matrix}
	\right) 
	= 
	\left(
	\begin{matrix}
	k_{11} & k_{12} & k_{13} & k_{14} \\
	k_{21} & k_{22} & k_{23} & k_{24} \\
	k_{31} & k_{32} & k_{33} & k_{34} \\
	k_{41} & k_{42} & k_{43} & k_{44} \\
	\end{matrix}
	\right) 
	\left(
	\begin{matrix}
	P_N \\ M_N \\ E_N \\ S_N
	\end{matrix}
	\right) 
	$$
	Specifically, $P$ is the negative effect of improper policy issued by the government, $M$ is the decrease of security apparatus, $E$ is economic declining, and $S$ is social unrest. Assume coefficient matrix is $K$, then the function becomes
	$$
	\left(
	\begin{matrix}
	P_{N+1} \\ M_{N+1} \\ E_{N+1} \\ S_{N+1}
	\end{matrix}
	\right) 
	= 
	K
	\left(
	\begin{matrix}
	P_N \\ M_N \\ E_N \\ S_N
	\end{matrix}
	\right) 
	$$
	
	\subsection{Modified Markov Model including climate change}
	
	We can find in the schema (\textbf{Figure 1}) that climate change has direct influence on \textbf{policy factor}, \textbf{economy factor} and \textbf{social factor}, which indicates that there are several modifications to make. Since only the \textbf{military factor} is not affected, the modified prediction function should be as follow
	
	$$
	\left(
	\begin{matrix}
	P_{N+1} \\ M_{N+1} \\ E_{N+1} \\ S_{N+1}
	\end{matrix}
	\right) 
	= 
	K 
	\left(
	\begin{matrix}
	P_N \\ M_N \\ E_N \\ S_N
	\end{matrix}
	\right) 
	+
	K_c
	C
	, \quad
	K_c = 
	\left(
	\begin{matrix}
	k_{CP} \\ {0} \\ k_{CE} \\ k_{CS}
	\end{matrix}
	\right)
	$$
	$CP$ is the influence , There may be doubts whether \textbf{military factor} will be affected. As a matter of fact, \textbf{military factor} will change, but not on the direct effect of climate change. The indirect influence from climate change will appear while running the modified prediction function. For example, catastrophe causes social instability, then the military level changes, which might take some time for strategic planning.
	
	
	\subsection{Estimation for some parameters}
	Three effects of climate change (natural disasters, sea-level rise, and increasing resource scarcity) are frequently assumed to lead to loss of livelihood, economic decline, and increased insecurity either directly or through forced migration. Interacting with poor governance, societal inequalities, and a bad neighborhood, these factors in turn may promote
	political and economic instability, social fragmentation, migration, and inappropriate responses from governments. Eventually this produces increased motivation for instigating violence as well as improved opportunities for mobilization.[2]
	
	Events related to climate may cause a huge influence on the society, since they will impact the provisionment, conditions of countries and communities, and acquisition of clean water and energy. For example, according to a report from the Climate Action Network of Australia, climate change may decrease the precipitation of prairies, which may lead to a $15\%$ decline in grass productivity. In turn, it could further result in a $12\%$ drop in average weight of cattle, which decreases the beef supply. Under such condition, milk yield of cattle will decline by $30\%$, and new pests will spread in the areas. Furthermore, such condition will cause a $10\%$ decline of drinking water. According to the model of upcoming changing conditions, in the next 15 - 30 years, such situation may happen in a number of grain production bases around the world, which will result in an enormous impact on the economy and society to a large extent. 
	
	In addition, A long line of research links hot temperatures to individual aggression, including violent crime and riots. The simple scarcity (neo-Malthusian) model of conflict assumes that if climate change results in a reduction in essential resources for livelihood, such as food or water, those affected by the increasing scarcity may start fighting over the remaining resources. Alternatively, people may be forced to leave the area and create new scarcities when they encroach on the territory of other people who may also be resource-constrained. Barnett and Adger (2007) review a broad range of studies of both of these effects, focusing particularly on countries where a large majority of the population is still dependent on employment in the primary sector. If climate change results in reduced rainfall and higher temperature that jointly causes droughts, and reduced access to the natural capital that sustains livelihoods, poverty will be more widespread and the potential for conflict greater.[2]
	
	\begin{table}[htbp]
		\renewcommand\arraystretch{1.5}
		\footnotesize
		\centering
		\begin{tabular}{m{3.8cm}<{\centering}|m{4.8cm}<{\centering}|m{4.8cm}<{\centering}}
			\whline
			\textbf{Hypothesis}&\textbf{$P_N$}&\textbf{$M_N$}\\
			\whline
			\textbf{Abnormal precipitation}& 2 support, 1 none, 1 opposite &1 support, 2 none, 1 opposite\\
			
			\textbf{Abnormal temperature}&1 support, 2 none, 1 opposite&1 support, 1 none, 1 opposite\\
			
			\textbf{Natural disasters}&2 support, 1 opposite&1 none\\
			
			\textbf{Sea level rising}&2 support, 1 none, 1 opposite&1 support, 1 none, 1 opposite\\
			
			\textbf{Energy deficiency}&2 support, 1 opposite&3 none\\
			
			\textbf{Less vegetation}&2 none&1 none\\
			
			\shline
			\textbf{Hypothesis}&\textbf{$E_N$}&\textbf{$S_N$}\\
			\whline
			\textbf{Abnormal precipitation}& 3 support, 1 some support &2 support, 1 opposite\\
			
			\textbf{Abnormal temperature}&4 support, 1 none, 2 opposite&2 support, 1 none, 1 opposite\\
			
			\textbf{Natural disasters}&6 support, 1 opposite&6 support, 2 opposite\\
			
			\textbf{Sea level rising}&3 support, 1 none, 1 opposite&2 none\\
			
			\textbf{Energy deficiency}&2 support, 1 opposite&2 support, 1 none, 1 opposite\\
			
			\textbf{Less vegetation}&3 support, 1 none, 1 opposite&2 support, 1 opposite\\
			\shline
		\end{tabular}
		\caption{Influence of climate change - Quantitative study}\label{tab:Influence of climate change - Quantitative study}
	\end{table}
	
	\begin{table}[htbp]
		\renewcommand\arraystretch{1.5}
		\footnotesize
		\centering
		\begin{tabular}{m{2.7cm}<{\centering}|m{5cm}<{\centering}|m{5cm}<{\centering}}
			\whline
			\textbf{Hypothesis}&\textbf{$P_N$}&\textbf{$M_N$}\\
			\whline
			\textbf{$P_N$}&3 support &2 support, 1 none, 2 opposite\\
			
			\textbf{$M_N$}&2 support, 1 none, 2 opposite&4 support\\
			
			\textbf{$E_N$}&4 support, 1 none&1 support, 1 opposite\\
			
			\textbf{$S_N$}&1 support, 1 none&2 support, 1 none, 1 opposite\\
			\shline
			\textbf{Hypothesis}&\textbf{$E_N$}&\textbf{$S_N$}\\
			\whline
			\textbf{$P_N$}& 3 support, 1 none, 1 opposite & 1 support, 1 none\\
			
			\textbf{$M_N$}&1 support, 2 none&1 support, 1 none\\
			
			\textbf{$E_N$}&3 support&1 support, 2 none\\
			
			\textbf{$S_N$}&1 support, 2 none&1 support\\
			\shline
		\end{tabular}
		\caption{Interplay of different factors - Quantitative study}\label{tab:Interplay of different factors - Quantitative study}
	\end{table}
	
	We apply quantitative analysis on the influence of climate change (\textbf{Table 3}) and interplay of other factors (\textbf{Table 4}). The figures in each cell denote the number of studies that support, do not support, or contradict the proposed hypothesis. The total number of
	reviewed studies per outcome is computed and will be used for the computation of parameters. To be more specific, for each cell we need to calculate an parameter that represents the influence of the row factor on the column factor. If there is a study that supports the relation, we increase the outcome by 1; if there is a study that is opposite to the relation, we decrease it by 1. Then we calculates the sum of each row, and consider the proportion of each cell in the sum as the parameter of that cell. 
	
	Finally ,we attain matrix $K$ and $K_c$ as follows
	\begin{spacing}{1.5}
		$$
		K = 
		\left(
		\begin{matrix}
		\frac{1}{2} & 0 & \frac{1}{3} & \frac{1}{6} \\
		0 & \frac{2}{3} & \frac{1}{6} & \frac{1}{6} \\
		\frac{1}{2} & 0 & \frac{3}{8} & \frac{1}{8} \\
		\frac{1}{4} &\frac{1}{4} & \frac{1}{4} & \frac{1}{4} \\
		\end{matrix}
		\right) 
		, \quad K_c = 
		\left(
		\begin{matrix}
		\frac{1}{8} \\ {0} \\ \frac{1}{2} \\ \frac{1}{4}
		\end{matrix}
		\right) 
		$$
	\end{spacing}
	Therefore, the final prediction function becomes
	\begin{spacing}{1.5}
		\begin{equation}
		\left(
		\begin{matrix}
		P_{N+1} \\ M_{N+1} \\ E_{N+1} \\ S_{N+1}
		\end{matrix}
		\right) 
		= 
		\left(
		\begin{matrix}
		\frac{1}{2} & 0 & \frac{1}{3} & \frac{1}{6} \\
		0 & \frac{2}{3} & \frac{1}{6} & \frac{1}{6} \\
		\frac{1}{2} & 0 & \frac{3}{8} & \frac{1}{8} \\
		\frac{1}{4} &\frac{1}{4} & \frac{1}{4} & \frac{1}{4} \\
		\end{matrix}
		\right) 
		\left(
		\begin{matrix}
		P_N \\ M_N \\ E_N \\ S_N
		\end{matrix}
		\right) 
		+
		\left(
		\begin{matrix}
		\frac{1}{8} \\ {0} \\ \frac{1}{2} \\ \frac{1}{4}
		\end{matrix}
		\right) 
		C
		\end{equation}
	\end{spacing}	
	
	\subsection{Obtain fragility standard}
	It makes no sense to draw up standards out of an imaginary country, so we choose to investigate several countries which are widely accepted as stable, vulnerable and fragile.
	We respectively define Egypt, China and Italy to be fragile, vulnerable and stable, then we do research in Fragile States Index database(see the appendix) to calculate the policy factor, military factor, economy factor and social factor. Our factors are gained by 
	\begin{spacing}{1.5}
		\begin{equation}
		\begin{aligned}
		P&=\frac{1}{2}PS+\frac{1}{2}HR\\
		M&=SA\\
		E&=\frac{1}{3}EC+\frac{1}{3}UD+\frac{1}{3}HF\\
		S&=\frac{1}{2}DP+\frac{1}{2}RI\\
		\end{aligned}
		\end{equation}
	\end{spacing}
	
	The results are as follows (\textbf{Table 5})
	
	\begin{table}[htbp]
		\renewcommand\arraystretch{1.5}
		\footnotesize
		\centering
		\begin{tabular}{m{2cm}<{\centering}|m{2cm}<{\centering}|m{2cm}<{\centering}|m{2cm}<{\centering}|m{2cm}<{\centering}}
			\whline
			&\textbf{$P$}&\textbf{$M$}&\textbf{$E$}&\textbf{$S$}\\
			\whline
			\textbf{Italy}& 2.35 & 4.5 & 3.43 & 4.9\\
			
			\textbf{China}& 7.1 & 5.9 & 5.53 & 5.8\\
			
			\textbf{Egypt}& 7.35 & 8.1 & 6.3 & 7.2\\
			
			\shline
		\end{tabular}
		\caption{Factors of Italy, China, and Egypt}\label{tab:Factors of Italy, China, and Egypt}
	\end{table}
	After getting these data, we use the evaluation function $(1)$ to obtain their fragility. (\textbf{Table 6})
	
	\begin{table}[htbp]
		\renewcommand\arraystretch{1.5}
		\footnotesize
		\centering
		\begin{tabular}{m{2cm}<{\centering}|m{5cm}<{\centering}}
			\whline
			&\textbf{$F$}\\
			\whline
			\textbf{Italy} & 15.1833\\
			
			\textbf{China} & 24.33\\
			
			\textbf{Egypt} & 28.95\\
			
			\shline
		\end{tabular}
		\caption{Fragility of Italy, China, and Egypt}\label{tab:Fragility  of Italy, China, and Egypt}
	\end{table}
	
	So we can now define stable as fragility below 15.18, vulnerable as fragility above 24.33 and fragile as fragility above 28.95.
	
	\subsection{The Impact of Climate Change}
	
	In this section, we use an example to further illustrate the impact of climate change. Applying our model, we measure the impact of climate change and analyze how it changes the fragility of a state. We first assume the change of climate is random, and compare the fragility of the country with and without the influence of climate change. Then we assume the climate change is always negative, and do the same comparison.
	
	As is shown in \textbf{Figure 2},  calculated by our model above, the red line represents the fragility of the country as time passes without the influence of climate change; the green line is the fragility with the indirect influence of climate change; the blue line is the fragility with the indirect and direct influence of climate change. According to the figure, the impact of climate change has both positive and negative side. 
	\begin{figure}[h]
		\small
		\centering
		\includegraphics[width=15cm]{figure3.jpg}
		\caption{Fragility-Year with random climate change} \label{fig:Fragility-Year with random climate change}
	\end{figure}

	In \textbf{Figure 3}, the implication of each line is same as what is illustrated above. It is clear that climate change increases fragility of the country. It will cause such consequence through direct impact and indirect means, which accords with the discussion in the previous chapters.
	\begin{figure}[h]
		\small
		\centering
		\includegraphics[width=15cm]{figure4.jpg}
		\caption{Fragility-Year with negative climate change} \label{fig:Fragility-Year with negative climate change}
	\end{figure}
	
	\section{Modifications for current process}
	After analyzing the flow of passengers and existing bottlenecks of the current process of security check, we propose several modifications in order to increase the throughput of passengers and reduce variance in waiting time. Since the calculation of the average waiting time depends on the values of $N_P$, $N_R$, $N_B$ and $N_X$ ( See \textbf{Figure 6}), we first modify the ratio of $N_P$, $N_R$, $N_B$ and $N_X$ so as to maximize the throughput, and on that basis we develop a dynamic optimization fuction to adjust the number of entrances as the flow of passengers entering the checkpoint changes, by which we can minimize variance in waiting time.  
	\begin{figure}[h]
		\small
		\centering
		\includegraphics[width=15cm]{figure6.png}
		\caption{Skecth of a security checkpoint} \label{fig:Skecth of a security checkpoint}
	\end{figure}
	\subsection{Modifications for increasing passenger throughput}
	We find three small modifications to increase passenger flow when the throughput gets saturated.
	\subsubsection{Change ratio of $N_X$ to $N_B$}
	We look into the data provided, and find that the time for people's properties getting scanned is much more than the time costed on themselves. So, if we could change $N_X/N_B$ to a more suitable value, the time utilization ratio can be developed. We analyze the data and use Successive Minus to calculate the ratio of average time spent on properties and persons, then we find that
	$$t_X:t_B=2.2:1$$
	So, when we do not have a large number of screening-lanes, we can set 
	$$N_X:N_B=2:1$$
	That is to say, when every two people share one millimeter wave body scanner and use two screening belt, the waiting time reach its minimum value and the efficiency of Zone B gets maximum. There is an accurate definition for "waiting" here that one person stops waiting as soon as he(she) puts his(her) baggage on the X-ray belt.
	\subsubsection{Change ratio of $N_P$ to $N_R$}
	We have calculated the ratio of people choosing Pre-Check to people choosing Regular-Check in our basic model according to the data.
	$$\lambda_P:\lambda_R=3:2$$
	Generally, we can set the ratio to
	$$N_P:N_R=3:2$$
	\subsubsection{Change the ratio of $(N_P+N_R)$ to $N_X$}
	The modification above applied, the average time for passengers passing the screening-lanes is shortened, and the average flow for each lane is enlarged
	$$\widehat{\mu_B}=\frac{1}{\overline{t_B}'}=0.0714s^{-1}$$
	$\mu_A$ is still the same value. Thus, due to the stability equation
	$$(N_P+N_R)\mu_A=N_X\mu_B$$
	We can find when
	$$N_X=1.2(N_P+N_R)$$
	the whole system is stable and there will be no congestion inside the security checkpoint.
	The further modifications are all based on these three basic principles.
	\subsection{Modifications for reducing time variance}
	In this section, we apply the Queuing Theory Model to simulate the process of security check. According to the Queuing Theory Model, we can obtain a relationship between $W_q$ (average waiting time for every passenger) and $s$ (the number of entrances). In order to minimize the variance of $W_q$, the airport should adjust $s$ based on various flow of passengers entering the checkpoint, which is related to $\lambda$.
	
	According to the assumptions we have already proposed, the number of arriving passengers per unit time obey Poisson Distribution $\pi(\lambda)$, and time passing through the ID-check or screening area obey Exponential Distribution $E(\mu)$. Define $\rho=\frac{\lambda}{s\mu}$. From the Queuing Theory Model, we can obtain the following results:
	
	\begin{itemize}
		\item When the number of entrances is $s$, the probability $P_n(s)$ when there are $n$ passengers in the system  is
		\begin{equation}
		P_n(s)=
		\begin{cases}
		\frac{(s\rho)^n}{n!}P_0(s), &\mbox{$n=1,2,\ldots,s$}\\
		\frac{s^s\rho^n}{s!}P_0(s)=\rho^{n-s}P_s(s), &\mbox{$n=s+1,s+2,\ldots$}
		\end{cases}
		\end{equation}
		$$P_0(s)=[\sum_{n=0}^{s-1}\frac{(s\rho)^n}{n!}+\frac{(s\rho)^s}{s!}(\frac{1}{1-\rho})]^{-1}$$
		\item The average number of passengers waiting in queues $L_q$ is
		$$L_q=\sum_{n=s+1}^{\infty}(n-s)P_n(s)=\frac{(s\rho)^s\rho}{s!(1-\rho)^2}P_0(s)$$
		\item The average number of passengers in the checkpoint $L_s$ is 
		$$L_S=L_q+s\rho$$
		\item The average time waiting in queues $W_q$ is
		$$W_q=\frac{L_q}{\lambda}$$
		\item The average time spending in the checkpoint $W_s$ is
		$$W_s=\frac{L_s}{\lambda}$$
	\end{itemize}
	
	More specifically, since in the previous modifiations we rearranged the proportianal relationship between $N_P$ and $N_R$ so as to make sure the passengers uniformly entering every entrances. Therefore, we consider the whole checkpoint as a M/M/S system, where $s$ is the sum of $N_P$ and $N_R$, and $\lambda$ is the number of passengers  arriving at every entrance per unit time. Furthermore, since the total servce time is the sum of time in Zone A and Zone B, $\mu$ can be attained as follows.
	$$\mu=\frac{1}{\frac{1}{\mu_A}+\frac{1}{\mu_B}}=0.0433s^{-1}$$
	Then, we use Matlab to simulate our model. Firstly, we assign different values to $s$ from 2 to 15 and calculate the corresponding $\lambda$ on condition that $W_q$ keeps constant ($25s$). Results are shown in \textbf{Table 3}.
	
	\begin{table}[htbp]
		\centering
		\begin{tabular}{m{4cm}<{\centering}m{4cm}<{\centering}}
			\toprule
			\textbf{$s$}&\textbf{$\lambda\ (s^{-1})$}\\
			\midrule
			2&0.0866\\
			
			3&0.1299\\
			
			4&0.1732\\
			
			5&0.2156\\
			
			6&0.2598\\
			
			7&0.3031\\
			
			8&0.3464\\
			
			9&0.3897\\
			
			10&0.4330\\ 
			
			11&0.4763\\ 
			
			12&0.5196\\
			
			13&0.5629\\
			
			14&0.6062\\
			
			15&0.6495\\
			\bottomrule
		\end{tabular}
		\caption{$s-\lambda$}\label{tab:Data}
	\end{table}
	From the data above, it can be discovered that $s$ and $\lambda$ are directly propotional. Therefore, assuming that $s$ and $\lambda$ are continuous, we can figure out the fuctional relationship between $\lambda$ and $s$:
	$$s=\frac{\lambda}{0.0433}$$
	We notice that $\mu=0.0433s^{-1}$, so we can draw the conclution that
	$$s=\frac{\lambda}{\mu}$$
	Then, we apply different approaches to round $s$ to appropriate integers. See \textbf{Figure 7}.
	
	
	Moreover, we draw the fitting figure between $W_q$ and $\lambda$ in \textbf{Figure 8} in order to assess different  approaches. In \textbf{Figure 8}, (1) and (2) will lead to long waiting time under specific conditions, which is due to the unstable state of the system. Such unstability is the result of $\rho=\frac{\lambda}{s\mu}>1$. As for (3) and (4), such unstability can be avoided, but (3) brings higher cost. Thus, we apply the approach in (4) to round $s$. 
	\begin{figure}[H]
		\small
		\centering
		\includegraphics[width=14.5cm]{figure7.png}
		\caption{$s-\lambda$ of different approaches} \label{fig:7}
	\end{figure} 
	
	In conclution, as $\lambda$ changes, we set $s$ as $s=\lceil\frac{\lambda}{\mu}-0.5\rceil+1$, which means when the flow of passengers entering the checkpoint increase, the airport need to increase the number of entrances according to the relationship above. By this method, we can reduce variance in waiting time by $\Delta W_q \leq \pm19.4s$.
	\begin{figure}[H]
		\small
		\centering
		\includegraphics[width=14.5cm]{figure8.png}
		\caption{$W_q-\lambda$ of different approaches} \label{fig:8}
	\end{figure}
	\section{Sensitivity analysis on cultural diversity}
	Since passengers from different areas have their own customs and norms, their behaviors are different, which could possibly influence our model. In this chapter, we analyze three representative types of passengers and how their cultural diversities impact our model. In the end, we demonstrate how our model accommodate these differences. 
	\subsection{Entrances paralyzed due to cutting in lines}
	Passengers from some areas have the concept of prioritizing individual efficiency, so the probability of cutting in lines may increase. Meanwhile, if the dispute caused by cutting in lines can not be solved in time, it may lead to the temporary paralysis of an entrance.
	
	We assume that the number of lanes paralyzed obeys Poisson Distribution. Due to scarcity of concrete data, we assume that it obeys $\pi(0.01)$. We have
	$$P(S_{paralyzed}=k)=\frac{e^{-0.01}(0.01)^k}{k!}$$
	From the calculation we can attain that the probability of a entrance paralyzed is approximately 1\%, of two entrances paralyzed is about 0.005\%, and of more entrances paralyzed is almost 0. Assumptions above are appropriate for the time period of general passenger flow. Therefore, we analyze the sensitivity according to the situation that there are only one or two entrances paralyzed.
	
	Based on the previous optimization, we have already obtained that without the possibility of entrances paralyzed $W_q$ is about 20s, and no more than 40s .
	\subsubsection{One entrance paralyzed}
	From previous analysis, although the actual number of entrances is one less than the number we anticipate, it will result in the rapid increase of $W_q$. Assume that there is one entrance paralyzed, which means that passengers who are supposed to go through this entrance will wait in other entrances. The increase of other entrances should obey
	$$\Delta L_q=\frac{\lambda\Delta t}{s(s-1)}$$
	From what has been deduced before
	$$s=\lceil\frac{\lambda}{\mu}-0.5\rceil+1$$
	We have
	$$\frac{dL_q}{dt}=\frac{\lambda}{\lceil\frac{\lambda}{\mu}-0.5\rceil(\lceil\frac{\lambda}{\mu}-0.5\rceil +1)}$$
	\begin{figure}[h]
		\small
		\centering
		\includegraphics[width=10cm]{figure9.png}
		\caption{$\frac{dL_q}{dt}-\lambda$ when one entrance paralyzed} \label{fig:9}
	\end{figure}\\
	Under such rule, when there is one entrance paralyzed, the speed of increases of the number of passengers waiting in other entrances is shown in \textbf{Figure 9}. We assume that most of disputes will be solved in five minutes. From the figure we can estimate that the number of people increasing for every entrance is 3.9, and is at most 4, which is to say, $W_q$ at most will increase by 1.5 minutes with the probability of 1\%.
	\subsubsection{Two entrances paralyzed}
	Similarly, when there are two entrances paralyzed, the speed of increases of the number of passengers for each entrances is
	$$\frac{dL_q}{dt}=\frac{2\lambda}{\lceil\frac{\lambda}{\mu}-0.5\rceil(\lceil\frac{\lambda}{\mu}-0.5\rceil +1)}$$
	If the flow of passengers is so little that the entrances open are less than two, the paralysis hardly happens. As a consequence, we omit the situation when $\lambda<0.1$ in the \textbf{Figure 10}.
	\begin{figure}[h]
		\small
		\centering
		\includegraphics[width=10cm]{figure10.png}
		\caption{$\frac{dL_q}{dt}-\lambda$ when two entrances paralyzed} \label{fig:10}
	\end{figure}
	Under such situation, we also assume that a dispute can be solved in five minutes. Thus, the increasing number of passengers for each entrance is at most 12. Passengers can markedly feel the increase of passengers, but the waiting time for each individual still will increase no more than 5 minutes.
	
	The sensitivity analysis above is based on the distribution $\pi(0.01)$. If such paralyzing happens in a large scale, we do not need such model to analyze it. From our analysis, we can draw the conclusion that the influence of $W_q$ due to cutting in queues is small. To be more specific, the time will increase no more than 5 minutes.
	
	\subsection{Longer screening time}
	Passengers from different areas spend different time for security check. For instance, Egyptians and Jews are used to wearing kerchiefs while Americans are used to wearing simple jeans and T-shirt, which will absolutely result in different checking time.\\
	Assume that in this case 
	$$W_q'=1.2W_q$$ 
	Then 
	$$\mu'=\frac{1}{1.2}\mu=0.83\mu=0.0359$$ 
	After calculation, the changed functional relationship is $s=\frac{\lambda }{0.0359}$. \textbf{Figure 11} shows $s-\lambda$, and \textbf{Figure 12} shows $W_q-\lambda$.
	\begin{figure}[H]
		\small
		\centering
		\includegraphics[width=15cm]{figure11.png}
		\caption{$s-\lambda$ with longer screening time} \label{fig:11}
	\end{figure}
	\begin{figure}[h]
		\small
		\centering
		\includegraphics[width=15cm]{figure12.png}
		\caption{$W_q-\lambda$ with longer screening time} \label{fig:12}
	\end{figure}
	In this case, $\Delta W_q\leq \pm 24.0s$ when $W_q$ is constant (25s). Campared with the previous result $\Delta W_q\leq \pm 19.4s$, we can conclude that longer screening time due to different dressing has little impact on our model.
	
	\subsection{More passengers getting additional screening}
	Due to the culture discrepancies, passengers from different areas may have different understanding of definitions of contrabands. It is possible that some items are not allowed to take through the security checkpoint in some areas while such items can be safely carried in other areas. Furthermore, different countries have different norms of limited power of specific electronic equipments. Therefore, such discrepancies may lead to the increase of the probability of passengers from specific areas getting additional searching, which will decrease the efficiency of the whole system.
	
	In the previous chapter, we assume that the proportion of passengers getting additional screening is a small constant $\eta$ whose influence on the whole system can be ignored. Now considering the influences of culture discrepancies, we assume that $\eta=0.02$ which is large enough to affect the throughput of passengers. Assume that the additional searching time also obey Exponential Distribution $E(\mu_D)$, and $\mu_D=\frac{1}{10}\mu_B$.
	Then we have
	$$\mu'=\frac{1}{\frac{\eta}{\mu_D}+\frac{1}{\mu_A}+\frac{1}{\mu_B}}=\frac{1}{\frac{10\eta}{\mu_B}+\frac{1}{\mu}}$$
	Similarly, we show the relationship between $s-\lambda$ and $W_q-\lambda$ in \textbf{Figure 13} and \textbf{Figure 14}.
	\begin{figure}[H]
		\small
		\centering
		\includegraphics[width=15cm]{figure13.png}
		\caption{$s-\lambda$ with more passengers in Zone D} \label{fig:13}
	\end{figure}
	\begin{figure}[H]
		\small
		\centering
		\includegraphics[width=15cm]{figure14.png}
		\caption{$W_q-\lambda$ with more passengers in Zone D} \label{fig:14}
	\end{figure}
	When $W_q=25s$, we can calculate $\Delta W_q’\leq\pm23.9$. Campared with the previous result $\Delta W_q\leq \pm 19.4s$, we can safely draw theconclusion that more passengers in Zone D due to obsure perceptions of contraband have little impact on our model.
	\section{Policy and recommendations}
	\begin{itemize}
		\item Based on the model established primitively, the speed of document check is twice over that of screening, as a consequence, the screening process contributes most to the appearance of bottlenecks. Therefore, the airport should dispatch more skilled TSA officers to the screening area and expedite the speed of screening.
		\item Rationally arrange the number of entrances and lanes. From the optimization we have made, the best arrangement is shown below
		$$N_X:N_B=2:1$$
		$$\lambda_P:\lambda_R=3:2$$
		$$N_X=1.2(N_P+N_R)$$
		\item Monitor the flow of passengers in real time, and arrange the number of entrances according to the acquired values in order to reduce the cost on the basis of rational efficiency. The optimal result we obtain according to data we have is
		$$s=\lceil\frac{\lambda}{\mu}-0.5\rceil+1$$
		\item Arrange different number of TSA officers and security staffs in different security checkpoints in order to adjust the average waiting time of passengers. For instance, in the checkpoints with more queue-jumpers, we should arrange more staffs to insure that disputes caused by cutting will not continue for a long period. In the sensitivity analysis, the increasing velocity of queues satisfy that
		$$\frac{dL_q}{dt}=\frac{\lambda s_{paralyzed}}{s(s-s_{paralyzed})}$$
		We assumed that all kinds of disputes will be solved in five minutes, and if such time increases, the queue will be so long that passengers may suffer from anxiety.
		\item Promote Pre-Check service, and increase the number of Pre-Check entrances, which will provide convenience to passengers and save the checking time. It is mentioned that 45\% of passengers enroll in Pre-Check, but since Pre-Check passengers tend to travel more frequently by plane, such proportion cannot be the criterion when deciding the number of Pre-Check entrances and regular entrances.
		\item Establish a baggage buffer that used to temporarily store the baggage that has been checked. Since in our optimization each lane for body screening is accompanied with two lanes for baggage screening, which will increase the probability of baggage checked faster than passengers, such buffer can prevent the occurrence of congestion and improve the security level of baggage.
		\item In the sensitivity analysis, we find that culture discrepancies will not impact greatly the queuing time, and the existing congestion mainly due to the insufficiency of entrances. Therefore, we propose that if cost permitting, the airport should open more entrances and disperse entrances as much as possible in order to disperse passengers, which will relief the anxiety of them.
	\end{itemize}
	\section{Strengths and weaknesses}
	\subsection{Strengths}
	\begin{itemize}
		\item Our model separates the whole process of security check into ID-check part and screening part, and independently calculates the corresponding queuing conditions, which avoid the possible errors due to the complicated model.
		\item We estimate the arrival rate for each passenger and the checking time for each TSA officer, and assume that the former obeys Poisson Distribution and the latter obeys Exponential Distribution, which conform to the general principles.
		\item We apply the moment estimation to estimate relevant parameters of the system so as to obtain the basic model.
		\item We apply the Queuing Theory, and consider the average waiting time and the flow of passengers as our principal evaluation indexes, which correspond with profits of airports and passengers.
		\item When analyzing the sensitivity and improving the current model, we consider the difference of average waiting time and the maximal flow of passengers as our principal evaluation indexes, which guarantee that the airport can provide every passengers with the approximately same service with the maximal efficiency.
	\end{itemize}
	\subsection{Weaknesses}
	\begin{itemize}
		\item Since we do not have enough data, we have to estimate related parameters using moment estimation of data given. The error can be large when the data size is small.
		\item The bound between service time and waiting time is not explicit enough. We assume that when passengers put their baggage on the belt, they finish waiting and service time begins, but actually perhaps passengers consider the time between the moment they put down their baggage and the moment they receive body screening is also the waiting time.\item Since we have less estimation of costs and utilization of space, our model considers efficiency and flying experience of passengers as the principal indexes. In fact, we should modify our model according to actual costs of airport construction.
		\item We did not take the scale of the airport into account, and modeling based on the given data. If the scale of the airport is large enough, we need to rewrite some of our codes to obtain a more graceful result.
	\end{itemize}
	
	\begin{thebibliography}{99}
		\bibitem{1}\url{https://en.wikipedia.org/wiki/Fragile_state#Defining_fragile_states}
		\bibitem{2}Schwartz, P. and Randall, D. “An Abrupt Climate Change Scenario and Its Implications for United States National Security”, October 2003.
		\bibitem{3}Clara V. Marin, Colin G. Drury, Rajan Batta, Li Lin. Server Adaptation in an Airport Security System Queue[J].The OR Society, 2007, Vol.20:22-31.
		\bibitem{4}Zeng Junjie. Optimization and Configuration of Airport Security. Wide Angle, 2009:173-174.
		\bibitem{5}MI Kamien,NL Schwartz. Dynamic Optimization: The Calculus of Variations and Optimal Control in Economics \& Management.North Holland, 1981, 31(1):1252-1257.
		\bibitem{6}Zhang Guofen, Huang B Q, Zhang C Y. Probability Theory, Mathematical Statistics and Stochastic Process. Zhejiang University Press, 2011.
		\bibitem{7}Ivo Adan, Jacques Resing. Queueing Systems.Department of Mathematics and Computing Science Eindhoven University of Technology, P.O. Box 513, 5600 MB Eindhoven, The Netherlands, 2015.
		\bibitem{8}Queueing theory (https://en.wikipedia.org/wiki/Queueing\_theory).
	\end{thebibliography}
	
	\begin{appendices}
		
		\section{First appendix}
		Here are data  about how passengers proceed through each step of the security screening process. \\
		\begin{table}[htbp]
			\centering
			\begin{tabular}{m{1.3cm}<{\centering}m{1.3cm}<{\centering}m{1.3cm}<{\centering}m{1.3cm}<{\centering}m{1.3cm}<{\centering}m{1.3cm}<{\centering}m{1.3cm}<{\centering}m{1.3cm}<{\centering}}
				\toprule
				\textbf{A}&\textbf{B}&\textbf{C}&\textbf{D}&\textbf{E}&\textbf{F}&\textbf{G}&\textbf{H}\\
				\midrule					
				00:00.&	00:00.&	00:08.&	00:15.&	00:09.&	00:03.&	00:00.&	0:48\\
				00:11.&	00:09.&	00:05.&	00:12.&	00:20.&	00:06.&	00:02.&	0:45\\
				00:13.&	00:10.&	00:11.&	00:15.&	00:33.&	00:07.&	00:03.&	0:28\\
				00:14.&	00:11.&	00:10.&	00:20.&	00:36.&	00:09.&	00:11.&	0:25\\
				00:15.&	00:13.&	00:09.&	00:08.&	00:43.&	00:20.&		&	0:22\\
				00:25.&	00:14.&	00:09.&	00:08.&	00:52.&	00:22.&		&	0:24\\
				00:34.&	00:16.&	00:13.&	00:11.&	01:05.&	00:25.&		&	0:17\\
				00:54.&	00:27.&	00:15.&		&	01:16.&	00:41.&		&	0:33\\
				00:56.&	00:57.&	00:12.&		&	01:25.&	01:07.&		&	0:08\\
				01:15.&	01:21.&		&		&	01:32.&	01:09.&		&	0:10\\
				01:16.&	02:16.&		&		&	01:43.&	01:18.&		&	0:26\\
				01:44.&	02:24.&		&		&	01:58.&		&		&	0:32\\
				01:49.&	02:27.&		&		&	02:09.&		&		&	0:21\\
				02:04.&	02:43.&		&		&	02:26.&		&		&	0:37\\
				02:26.&	02:44.&		&		&	02:40.&		&		&	1:08\\
				
				\bottomrule
			\end{tabular}
		\end{table}
		\begin{table}[htbp]
			\centering
			\begin{tabular}{m{1.3cm}<{\centering}m{1.3cm}<{\centering}m{1.3cm}<{\centering}m{1.3cm}<{\centering}m{1.3cm}<{\centering}m{1.3cm}<{\centering}m{1.3cm}<{\centering}m{1.3cm}<{\centering}}
				\toprule
				\textbf{A}&\textbf{B}&\textbf{C}&\textbf{D}&\textbf{E}&\textbf{F}&\textbf{G}&\textbf{H}\\
				\midrule
				02:54.&	03:06.&		&		&	02:47.&		&		&	0:40\\
				02:58.&	03:13.&		&		&	02:54.&		&		&	0:18\\
				02:59.&	03:15.&		&		&	03:06.&		&		&	0:26\\
				03:01.&	03:27.&		&		&	03:25.&		&		&	0:08\\
				03:06.&	03:29.&		&		&	03:34.&		&		&	0:21\\
				03:09.&	04:27.&		&		&	04:11.&		&		&	0:23\\
				03:10.&	04:42.&		&		&	04:22.&		&		&	0:28\\
				03:20.&	04:48.&		&		&	04:35.&		&		&	0:50\\
				03:22.&	05:00.&		&		&	04:47.&		&		&	0:28\\
				03:34.&	05:09.&		&		&	05:00.&		&		&	0:48\\
				03:53.&	05:40.&		&		&	05:11.&		&		&	0:28\\
				04:17.&	05:57.&		&		&	05:18.&		&		&	0:36\\
				04:35.&	06:00.&		&		&	05:26.&		&		&	0:27\\
				04:36.&	06:20.&		&		&	05:41.&		&		&	0:05\\
				04:37.&	07:39.&		&		&	05:48.&&&\\		
				04:38.&	07:51.&		&		&	05:59.&&&\\			
				05:06.&	08:01.&		&		&	06:09.&&&\\			
				05:13.&	08:02.&		&		&	06:36.&&&\\			
				05:37.&	08:04.&		&		&	06:45.&&&\\			
				05:42.&	08:06.&		&		&	06:54.&&&\\		
				05:46.&	08:18.&		&		&	07:05.&&&\\			
				06:03.&	08:20.&		&		&	07:15.&&&\\			
				06:06.&	08:21.&		&		&	07:24.&&&\\			
				06:11.&	08:22.&		&		&	07:36.&&&\\			
				06:46.&	08:33.&		&		&	07:43.&&&\\		
				06:47.&	08:46.&&&&&&\\						
				06:49.&	08:54.&&&&&&\\						
				06:50.&	08:56.&&&&&&\\						
				07:06.&	09:13.&&&&&&\\						
				07:07.&	09:34.&&&&&&\\						
				07:27.&	09:53.&&&&&&\\						
				07:36.&	09:56.&&&&&&\\	
				07:41.&&&&&&&\\							
				07:41.&&&&&&&\\						
				07:42.&&&&&&&\\							
				07:47.&&&&&&&\\							
				08:16.&&&&&&&\\							
				08:23.&&&&&&&\\							
				08:25.&&&&&&&\\							
				08:26.&&&&&&&\\							
				08:27.&&&&&&&\\							
				08:43.&&&&&&&\\							
				08:44.&&&&&&&\\							
				\bottomrule
			\end{tabular}
		\end{table}
		
		\begin{table}[htbp]
			\centering
			\begin{tabular}{m{1.2cm}<{\centering}|m{3cm}<{\centering}|m{9cm}<{\centering}}
				\whline
				\textbf{Symbol}&\textbf{Process}&\textbf{Notes}\\
				\hline 
				A&TSA PreCheck Arrival Times&	Airport checkpoint recoding individuals entering the pre-check queue.\\
				\hline 
				B&Regular Arrival Times	&Airport checkpoint recoding individuals entering the regular queue.\\
				\hline 
				C&ID Check TSA officer 1	&The time the arrival of the passenger to the ID check station until the TSA officer calls the next passenger forward.\\
				\hline 
				D&ID Check TSA officer 2	&Same as column C, but for a different TSA officer.\\
				\hline 
				E&mm wave scan times	&Time stamps as passenger exited the milimeter wave scanner.\\
				\hline 
				F&X-Ray Scan Time 1	&Time stamps as bags exited the x-ray screening.\\
				\hline 
				G&X-Ray Scan Time 2	&Same as column F, but for a different TSA officer.\\
				\hline  
				H&Time to get scanned property	&Time it takes people from arriving at the belt to place items to be scanned, until they retrieved their items off the post-xray belt.\\
				\shline
			\end{tabular}
		\end{table}
		
		
		
		
		
		
		
		
		
		\section{Second appendix}
		Here are simulation programmes we used in our model as follow.
		\noindent \textbf{\textcolor[rgb]{0.98,0.00,0.00}{Draw the figure of $f_{omax}$ and $N_L/N_P$}}
%		\lstinputlisting[language=Matlab]{./code/fomax_NL_NP.m}
%		\textbf{\textcolor[rgb]{0.98,0.00,0.00}{Find the zero point of the function $test$ }}
%		\lstinputlisting[language=Matlab]{./code/drive_test.m}
%		\textbf{\textcolor[rgb]{0.98,0.00,0.00}{Calculate $W_q$ for every specific $s$}}
%		\lstinputlisting[language=Matlab]{./code/test.m}
%		\textbf{\textcolor[rgb]{0.98,0.00,0.00}{Draw figures of $s$ and $\lambda$ when approaches to rounding $s$ are different}}
%		\lstinputlisting[language=Matlab]{./code/s_lambda1.m}
%		\textbf{\textcolor[rgb]{0.98,0.00,0.00}{Draw figures of $W_q$ and $\lambda$ when approaches to rounding $s$ are different}}
%		\lstinputlisting[language=Matlab]{./code/Wq_lambda1.m}
%		\textbf{\textcolor[rgb]{0.98,0.00,0.00}{Draw figures of $s$ and $\lambda$ when screening time is longer}}
%		\lstinputlisting[language=Matlab]{./code/s_lambda2.m}
%		\textbf{\textcolor[rgb]{0.98,0.00,0.00}{Draw figures of $W_q$ and $\lambda$ when screening time is longer}}
%		\lstinputlisting[language=Matlab]{./code/Wq_lambda2.m}
%		\textbf{\textcolor[rgb]{0.98,0.00,0.00}{Draw figures of $s$ and $\lambda$ when there are more passengers entering Zone D}}
%		\lstinputlisting[language=Matlab]{./code/s_lambda3.m}
%		\textbf{\textcolor[rgb]{0.98,0.00,0.00}{Draw figures of $W_q$ and $\lambda$ when there are more passengers entering Zone D}}
%		\lstinputlisting[language=Matlab]{./code/Wq_lambda3.m}
%		\noindent \textbf{\textcolor[rgb]{0.98,0.00,0.00}{Draw the figure of $dW_q/dt-\lambda$ when there is one entrance paralyzed}}
%		\lstinputlisting[language=Matlab]{./code/OneEntranceParalyzed.m}
%		\textbf{\textcolor[rgb]{0.98,0.00,0.00}{Draw the figure of $dW_q/dt-\lambda$ when there are two entrances paralyzed }}
%		\lstinputlisting[language=Matlab]{./code/TwoEntranceParalyzed.m}
	\end{appendices}
\end{document}

%% 
%% This work consists of these files mcmthesis.dtx,
%%                                   figures/ and
%%                                   code/,
%% and the derived files             mcmthesis.cls,
%%                                   mcmthesis-demo.tex,
%%                                   README,
%%                                   LICENSE,
%%                                   mcmthesis.pdf and
%%                                   mcmthesis-demo.pdf.
%%
%% End of file `mcmthesis-demo.tex'.